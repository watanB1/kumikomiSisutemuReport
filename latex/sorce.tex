\documentclass{jarticle}
\usepackage[utf8]{inputenc}

\usepackage{listings}
\usepackage{xcolor}

%New colors defined below
\definecolor{codegreen}{rgb}{0,0.6,0}
\definecolor{codegray}{rgb}{0.5,0.5,0.5}
\definecolor{codepurple}{rgb}{0.58,0,0.82}
\definecolor{backcolour}{rgb}{0.95,0.95,0.92}

%Code listing style named "mystyle"
\lstdefinestyle{mystyle}{
  backgroundcolor=\color{backcolour},   commentstyle=\color{codegreen},
  keywordstyle=\color{magenta},
  numberstyle=\tiny\color{codegray},
  stringstyle=\color{codepurple},
  basicstyle=\ttfamily\footnotesize,
  breakatwhitespace=false,         
  breaklines=true,                 
  captionpos=t,                    
  keepspaces=true,                 
  numbers=left,                    
  numbersep=5pt,                  
  showspaces=false,                
  showstringspaces=false,
  showtabs=false,                  
  tabsize=2
}

%"mystyle" code listing set
%\lstset{style=mystyle}
\lstset{
	%プログラム言語(複数の言語に対応,C,C++も可)
 	language = HTML,
 	%背景色と透過度
 	backgroundcolor={\color[gray]{.90}},
 	%枠外に行った時の自動改行
 	breaklines = true,
 	%自動開業後のインデント量(デフォルトでは20[pt])	
 	breakindent = 10pt,
 	%標準の書体
 	basicstyle = \ttfamily\scriptsize,
 	%basicstyle = {\small}
 	%コメントの書体
 	commentstyle = {\itshape \color[cmyk]{1,0.4,1,0}},
 	%関数名等の色の設定
 	classoffset = 0,
 	%キーワード(int, ifなど)の書体
 	keywordstyle = {\bfseries \color[cmyk]{0,1,0,0}},
 	%""で囲まれたなどの"文字"の書体
 	stringstyle = {\ttfamily \color[rgb]{0,0,1}},
 	%枠 "t"は上に線を記載, "T"は上に二重線を記載
	%他オプション:leftline,topline,bottomline,lines,single,shadowbox
 	frame = TBrl,
 	%frameまでの間隔(行番号とプログラムの間)
 	framesep = 5pt,
 	%行番号の位置
 	numbers = left,
	%行番号の間隔
 	stepnumber = 1,
	%右マージン
 	%xrightmargin=0zw,
 	%左マージン
	%xleftmargin=3zw,
	%行番号の書体
 	numberstyle = \tiny,
	%タブの大きさ
 	tabsize = 4,
 	%キャプションの場所("tb"ならば上下両方に記載)
 	captionpos = t
}

\title{Code Listing}
\date{ }

\begin{document}

\maketitle

\section{Code examples}

プログラムはListing\ref{kadai3}のようになる。
%Python code highlighting
\begin{lstlisting}[language=HTML, caption=HTMLのプログラム例,label=kadai3]
<!DOCTYPE HTML>
<html>
<head>
    <meta charset="utf-8">
    <meta name="viewport" content="width=device-width, initial-scale=1, maximum-scale=1, user-scalable=no">
    <meta http-equiv="Content-Security-Policy" content="default-src * data: gap: content: https://ssl.gstatic.com; style-src * 'unsafe-inline'; script-src * 'unsafe-inline' 'unsafe-eval'">
    <script src="components/loader.js"></script>
    <link rel="stylesheet" href="components/loader.css">
    <link rel="stylesheet" href="css/style.css">
    <script>
    </script>
</head>
<body>
  <br />
    This is a template for Monaca app for プロジェクト実習実験.
    <input id="x"/>
    <input id="y"/>
    <button onclick="setText();">Set</button>
    <div id="text">This is text.</div>
</body>
</html>
\end{lstlisting}

%\lstlistoflistings

\end{document}
